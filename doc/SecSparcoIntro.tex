\chapter{Introduction}

The focus of the first version of \sparco{} was to provide a unified
collection of test problems for compressed sensing. Each problem is of
the form
\[
b = MBx_0 + e,\quad \mbox{or}\quad b = Ms + e,
\]
where $b$ is the observed vector, $M$ is a measurement matrix, $s$ is
a signal which can be sparsely represented as the product $Bx_0$,
where $x_0$ is sparse, and $e$ is additive noise. Especially in the
earlier years of compressed sensing (or atomic decomposition), the
sparsity matrix $B$ consisted of a concatenation of sparsity bases
such as the discrete Fourier, cosine, and wavelet transforms. One
salient feature of these transforms is that there exist ways to
evaluate matrix-vector products with these bases, without ever having
to form the underlying matrix explicitly. This leads to fast
multiplication times as well as a high level of scalability in matrix
dimensions. When combining such operators into a compound operator it
is then desirable to preserve these properties. One way to do this is
to write special code for each distinct $B$. This is very time
consuming and a more general framework was set up to implement all
operators. 

After the release of \sparco{} it became increasingly clear that the
operator framework was very useful on its own, which then led to the
development of the \sparco{} operator toolbox (\spot). The aim of the
toolbox is to make operators feel like matrices and allowing many of
the standard methods to be applied to them. The result is a
comprehensive set of operators and methods on operators. Each operator
implements multiplication by itself and by its complex conjugate (or
transpose for real matrices).

%That means that when $A$ and $B$ are some \sparco{} operators, we can
%write
%\begin{codeblock}
%y = A * x;
%z = B'* y;
%\end{codeblock}
%Like with matrices it would also be perfectly fine to write things as
%\begin{codeblock}
%C = B'* A; % Create compound operator
%z = C * x;
%D = [A, B];
%\end{codeblock}

% - Develop sparco, set of test problems for sparse recovery
% - Recognized the need for flexible operators
% - Initially this was done using function handles alone
% - Since then, the flexible set of operators itself became more popular
% - Developed now into full set of operators, still based on function
%   handles to easy of extension, but now wrapped in a Sparco class
% - The aim was to make operators feel like matrices and allowing many
%   of the standard methods to be applied to them.
% - The result (The \sparco{} operator toolbox) is a comprehensive set
%   of operators and methods.


% Element specific operations are slow; and operators such at .*
% cannot be implemented without losing the implicit nature of operators.

%%% Local Variables: 
%%% mode: latex
%%% TeX-master: "manual"
%%% End: 
